\documentclass{beamer}

\usepackage{amsmath}
\usepackage{listings}
\usepackage{inconsolata}
\usepackage{color}
\usepackage[T1]{fontenc}

\definecolor{bluekeywords}{rgb}{0.13,0.13,1}
\definecolor{greencomments}{rgb}{0,0.5,0}
\definecolor{redstrings}{rgb}{0.9,0,0}

\lstset{language=Java,
  showspaces=false,
  showtabs=false,
  breaklines=true,
  showstringspaces=false,
  breakatwhitespace=true,
  commentstyle=\color{greencomments},
  keywordstyle=\color{bluekeywords},
  stringstyle=\color{redstrings},
  basicstyle=\ttfamily
  }

\lstdefinestyle{highlight}{
  keywordstyle=\color{bluekeywords},
  commentstyle=\color{greencomments},
  basicstyle=\ttfamily,
}

\lstdefinestyle{allwhite}{
  keywordstyle=\color{white},
  commentstyle=\color{white},
  basicstyle=\ttfamily,
}

\lstdefinestyle{base}{
  language={Java},
  basicstyle={\color{black!40}\ttfamily},
  keywordstyle=\color{bluekeywords!40},
  commentstyle=\color{greencomments!40},
  stringstyle=\color{redstrings!40},
  moredelim=**[is][\only<+>{\color{black}\lstset{style=highlight}}]{@}{@},
}

\lstdefinestyle{hiddencode}{
  language={Java},
  basicstyle={\color{black}\ttfamily},
  keywordstyle=\color{bluekeywords},
  commentstyle=\color{greencomments},
  stringstyle=\color{redstrings},
  moredelim=**[is][\only<+>{\color{white}\lstset{style=allwhite}}]{@}{@},
}

\lstdefinestyle{basenopause}{
  language={Java},
  basicstyle=\color{black},
  keywordstyle=\color{red},
  commentstyle=\color{green!60!black},
}

\title{Programming II: Attack of the Arrays}
\date{}

\begin{document}

\frame{\titlepage}

\begin{frame}[fragile]
\frametitle{Arrays}
\note{Of course, explain to students what's going on here. Make sure they understand why c is invalid (it has a double in an int array). Also note that the fifth bullet point, b[3], returns an error.}
Sometimes we want to store multiple values in one variable. This is called an array.
\pause
\medskip

This is one way to declare an array:
\begin{lstlisting}[style=base]
@int[] a@ = @{4, 5, 10, 1}@;
@double[] b = {4.4, 1.0, -2.1}@;

@int[] c = {4, 0, -1, 3.3}@;
\end{lstlisting}

\begin{itemize}
\item<6->a[0]
\item<7->a[1]
\item<8->b[1]
\item<9->b[2]
\item<10->b[3]
\item<11->a.length
\item<12->b.length
\end{itemize}
\end{frame}

\begin{frame}[fragile]
\frametitle{Array Example}
A function that returns the sum of the first and second elements of an int array:

\begin{lstlisting}[style=basenopause]
int sumFirstTwo(int[] nums) {
    return nums[0] + nums[1];
}
\end{lstlisting}
\begin{itemize}
\item<2->sumFirstTwo(\{1, 2, 3, 4, 5\})
\item<3->sumFirstTwo(\{0, 6, 3, 2\})
\item<4->sumFirstTwo(\{-6, 6, 12, 3\})
\item<5->sumFirstTwo(\{4\})
\end{itemize}
\end{frame}

\begin{frame}[fragile]
\frametitle{Array Example 2}
A function that returns the sum of the first and second elements of an int array, OR just the first element, if there is only one list entry.

\begin{lstlisting}[style=basenopause]
private int sumFirstOneOrTwo(int[] nums) {
    if (nums.length >= 2) {
        return nums[0] + nums[1];
    } else {
        return nums[0];
    }
}
\end{lstlisting}
\end{frame}

\begin{frame}[fragile]
\frametitle{Creating an array}
Given an int array $nums$ of length 3, return a new array with the elements in reverse order, so {1, 2, 3} becomes {3, 2, 1}. (codingbat problem)
\begin{lstlisting}[style=basenopause]
private int[] reverse3(int[] nums) {
    int[] a = new int[3];
    
    // what is a.length?
    
    a[0] = nums[2];
    a[1] = nums[1];
    a[2] = nums[0];
    
    return a;
}
\end{lstlisting}
\end{frame}

\begin{frame}[fragile]
\frametitle{Last element of an array}
\note{For this slide, I recommend opening up a text editor and putting up a few arrays. Ask students what the length of each array is, then ask what the index of the last element is.}
How can we get the last element of an array?

\begin{lstlisting}[style=hiddencode]
private int lastElement(int[] nums) {
    // ?????
    @return nums[nums.length - 1];@
}
\end{lstlisting}
\pause
\end{frame}

\begin{frame}[fragile]
\frametitle{Write a function: array edition}
Given an int array $nums$, write a function $firstLastEqual$ that returns true if the first and last array elements are equal. (modified codingbat problem)
\pause 
(do this problem in a separate text editor before proceeding)
\pause
\begin{lstlisting}
private boolean firstLastEqual(int[] nums) {
    return nums[0] == nums[nums.length-1];
}
\end{lstlisting}
\end{frame}

\begin{frame}
\frametitle{Assignment}
Go to the Java Array-1 section of CodingBat. Do all of the problems you can handle.
\end{frame}
\end{document}
